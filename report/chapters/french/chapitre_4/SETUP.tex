\documentclass{article}
\usepackage[utf8]{inputenc}
\usepackage{amsmath}
\usepackage{listings}
\usepackage{xcolor}

\lstdefinelanguage{CSS}{
    keywords={color, background-color, margin, padding, border},
    sensitive=false,
    morecomment=[l]{//},
    morestring=[b]",
}

\lstdefinelanguage{TypeScript}{
    keywords={import, from, class, extends, public, private, string, number, console, log, export, const, let, return, new},
    sensitive=true,
    morecomment=[l]{//},
    morecomment=[s]{/*}{*/},
    morestring=[b]",
    morestring=[b]',
    morestring=[b]`
}

\lstset{
    basicstyle=\ttfamily\footnotesize,
    keywordstyle=\color{blue}\bfseries,
    stringstyle=\color{red},
    commentstyle=\color{green!70!black},
    numbers=left,
    numberstyle=\tiny\color{gray},
    stepnumber=1,
    numbersep=10pt,
    backgroundcolor=\color{gray!10},
    frame=single,
    breaklines=true,
    showstringspaces=false,
    tabsize=2,
    captionpos=b
}

\title{Configuration d'un Projet Angular}
\author{}
\date{}

\begin{document}

\maketitle

\section*{Prérequis}
\subsection*{1. Node.js et npm (Node Package Manager)}
Angular nécessite Node.js et npm pour fonctionner. Si vous ne les avez pas encore installés, téléchargez et installez la version la plus récente de Node.js depuis \texttt{https://nodejs.org}.

Vous pouvez vérifier l'installation en exécutant les commandes suivantes dans votre terminal :
\begin{verbatim}
node -v
npm -v
\end{verbatim}

\subsection*{2. Angular CLI}
Angular CLI (Command Line Interface) est un outil de ligne de commande qui simplifie le processus de création, de gestion et de test des applications Angular.

Pour installer Angular CLI globalement, exécutez la commande suivante :
\begin{verbatim}
npm install -g @angular/cli
\end{verbatim}

Vérifiez l'installation de Angular CLI :
\begin{verbatim}
ng version
\end{verbatim}

\section*{Étape 1 : Créer un Nouveau Projet Angular}
\subsection*{1. Créer le projet}
Dans le terminal, allez dans le répertoire où vous souhaitez créer votre projet Angular. Par exemple :
\begin{verbatim}
cd /chemin/vers/dossier
\end{verbatim}

Utilisez la commande \texttt{ng new} pour créer un nouveau projet Angular :
\begin{verbatim}
ng new nom-de-votre-projet
\end{verbatim}

Cette commande va vous poser quelques questions :
\begin{itemize}
    \item Voulez-vous utiliser Angular Routing ? Si vous avez besoin de la gestion des routes, répondez \texttt{Yes}.
    \item Choisir un format de feuille de style : CSS, SCSS, Sass, Less, ou Stylus. Choisissez le format que vous préférez (CSS par défaut).
\end{itemize}

Après avoir répondu aux questions, Angular CLI va créer le projet dans un dossier avec le nom que vous avez spécifié.

\subsection*{2. Naviguer dans le Dossier du Projet}
Une fois le projet créé, allez dans le répertoire du projet :
\begin{verbatim}
cd nom-de-votre-projet
\end{verbatim}

\subsection*{3. Lancer le Serveur de Développement}
Pour vérifier que tout fonctionne bien, lancez le serveur de développement avec la commande suivante :
\begin{verbatim}
ng serve
\end{verbatim}

Cette commande va :
\begin{itemize}
    \item Compiler le projet.
    \item Lancer un serveur de développement sur \texttt{http://localhost:4200}.
\end{itemize}

Vous pouvez maintenant ouvrir votre navigateur et accéder à l'application Angular en naviguant vers \texttt{http://localhost:4200}. Si tout fonctionne correctement, vous verrez la page d'accueil Angular par défaut.

\section*{Étape 2 : Structure du Projet}
Un projet Angular généré par Angular CLI a la structure suivante :
\begin{verbatim}
nom-de-votre-projet/
├── e2e/                           # Tests end-to-end
├── node_modules/                  # Dépendances du projet
├── src/                           # Code source de l'application
│   ├── app/                       # Composants, services, etc.
│   ├── assets/                    # Fichiers statiques (images, etc.)
│   ├── environments/              # Fichiers de configuration pour différents environnements
│   ├── index.html                 # Point d'entrée de l'application
│   ├── main.ts                    # Point d'entrée de l'application (fichier TypeScript)
│   └── styles.css                 # Feuille de style globale
├── angular.json                   # Fichier de configuration du projet Angular
├── package.json                   # Dépendances et scripts npm
└── tsconfig.json                  # Configuration TypeScript
\end{verbatim}

\section*{Étape 3 : Ajouter un Composant}
1. Pour créer un nouveau composant, utilisez la commande suivante :
\begin{verbatim}
ng generate component nom-du-composant
\end{verbatim}
Ou la forme abrégée :
\begin{verbatim}
ng g c nom-du-composant
\end{verbatim}

Cette commande va générer les fichiers nécessaires dans le dossier \texttt{src/app/nom-du-composant} :
\begin{itemize}
    \item \texttt{nom-du-composant.component.ts} (le fichier TypeScript du composant)
    \item \texttt{nom-du-composant.component.html} (le template HTML du composant)
    \item \texttt{nom-du-composant.component.css} (les styles du composant)
    \item \texttt{nom-du-composant.component.spec.ts} (le fichier de test unitaire)
\end{itemize}

\section*{Étape 4 : Ajouter un Service}
Les services sont utilisés pour fournir des données et des fonctionnalités dans l'application Angular.

1. Pour générer un service, utilisez la commande :
\begin{verbatim}
ng generate service nom-du-service
\end{verbatim}
Ou la forme abrégée :
\begin{verbatim}
ng g s nom-du-service
\end{verbatim}

Cela va créer un fichier de service dans \texttt{src/app/nom-du-service.service.ts}.

\section*{Étape 5 : Ajouter des Routes}
Pour configurer des routes dans Angular, vous devez :
1. Créer un fichier \texttt{app-routing.module.ts} si ce n'est pas déjà fait.
2. Définir les routes et importer le \texttt{RouterModule}.

Exemple :
\begin{lstlisting}[language=TypeScript, caption={Using a service in a component}, label={lst:typescript-service-usage}]

import { NgModule } from '@angular/core';
import { RouterModule, Routes } from '@angular/router';
import { HomeComponent } from './home/home.component';
import { AboutComponent } from './about/about.component';

const routes: Routes = [
  { path: '', component: HomeComponent },
  { path: 'about', component: AboutComponent }
];

@NgModule({
  imports: [RouterModule.forRoot(routes)],
  exports: [RouterModule]
})
export class AppRoutingModule { }
\end{verbatim}

\section*{Étape 6 : Déployer l'Application}
Une fois votre application prête, vous pouvez la construire pour la production en exécutant la commande suivante :
\begin{verbatim}
ng build --prod
\end{lstlisting}

Cette commande va créer une version optimisée de l'application dans le dossier \texttt{dist/}.

Vous pouvez ensuite déployer les fichiers de ce dossier sur votre serveur de production ou une plateforme cloud comme \texttt{Firebase}, \texttt{AWS}, ou \texttt{Netlify}.

\end{document}
