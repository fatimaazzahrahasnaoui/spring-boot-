\documentclass{article}
\usepackage{amsmath}
\usepackage{graphicx}
\usepackage[most]{tcolorbox}
\usepackage{listings}
\usepackage{xcolor}

\lstdefinelanguage{CSS}{
    keywords={color, background-color, margin, padding, border},
    sensitive=false,
    morecomment=[l]{//},
    morestring=[b]",
}

\lstdefinelanguage{TypeScript}{
    keywords={import, from, class, extends, public, private, string, number, console, log, export, const, let, return, new},
    sensitive=true,
    morecomment=[l]{//},
    morecomment=[s]{/*}{*/},
    morestring=[b]",
    morestring=[b]',
    morestring=[b]`
}

\lstset{
    basicstyle=\ttfamily\footnotesize,
    keywordstyle=\color{blue}\bfseries,
    stringstyle=\color{red},
    commentstyle=\color{green!70!black},
    numbers=left,
    numberstyle=\tiny\color{gray},
    stepnumber=1,
    numbersep=10pt,
    backgroundcolor=\color{gray!10},
    frame=single,
    breaklines=true,
    showstringspaces=false,
    tabsize=2,
    captionpos=b
}
\title{Setting Up an Angular Project}
\author{}
\date{}

\begin{document}

\maketitle

\section*{Prerequisites}
\subsection*{1. Node.js and npm (Node Package Manager)}
Angular requires Node.js and npm to function. If you haven't installed them yet, download and install the latest version of Node.js from \texttt{https://nodejs.org}.

You can verify the installation by running the following commands in your terminal:
\begin{verbatim}
node -v
npm -v
\end{verbatim}

\subsection*{2. Angular CLI}
Angular CLI (Command Line Interface) is a tool that simplifies the process of creating, managing, and testing Angular applications.

To install Angular CLI globally, run the following command:
\begin{verbatim}
npm install -g @angular/cli
\end{verbatim}

Verify the installation of Angular CLI:
\begin{verbatim}
ng version
\end{verbatim}

\section*{Step 1: Create a New Angular Project}
\subsection*{1. Creating the Project}
In the terminal, navigate to the directory where you want to create your Angular project. For example:
\begin{verbatim}
cd /path/to/folder
\end{verbatim}

Use the \texttt{ng new} command to create a new Angular project:
\begin{verbatim}
ng new your-project-name
\end{verbatim}

This command will prompt you with a few questions:
\begin{itemize}
    \item Do you want to use Angular Routing? If you need routing, answer \texttt{Yes}.
    \item Choose a stylesheet format: CSS, SCSS, Sass, Less, or Stylus. Select your preferred format (CSS by default).
\end{itemize}

After answering these questions, Angular CLI will create the project in a folder named after the specified project name.

\subsection*{2. Navigate to the Project Folder}
Once the project is created, navigate to the project directory:
\begin{verbatim}
cd your-project-name
\end{verbatim}

\subsection*{3. Start the Development Server}
To ensure everything is working correctly, start the development server with the following command:
\begin{verbatim}
ng serve
\end{verbatim}

This command will:
\begin{itemize}
    \item Compile the project.
    \item Start a development server at \texttt{http://localhost:4200}.
\end{itemize}

You can now open your browser and access the Angular application by navigating to \texttt{http://localhost:4200}. If everything is set up correctly, you will see the default Angular homepage.

\section*{Step 2: Project Structure}
An Angular project generated by Angular CLI has the following structure:
\begin{verbatim}
your-project-name/
├── e2e/                           # End-to-end tests
├── node_modules/                  # Project dependencies
├── src/                           # Application source code
│   ├── app/                       # Components, services, etc.
│   ├── assets/                    # Static files (images, etc.)
│   ├── environments/              # Configuration files for different environments
│   ├── index.html                 # Application entry point
│   ├── main.ts                    # Application entry file (TypeScript)
│   └── styles.css                 # Global stylesheet
├── angular.json                   # Angular project configuration
├── package.json                   # npm dependencies and scripts
└── tsconfig.json                  # TypeScript configuration
\end{verbatim}

\section*{Step 3: Add a Component}
1. To create a new component, use the following command:
\begin{verbatim}
ng generate component component-name
\end{verbatim}
Or the shorthand:
\begin{verbatim}
ng g c component-name
\end{verbatim}

This command will generate the necessary files in the \texttt{src/app/component-name} folder:
\begin{itemize}
    \item \texttt{component-name.component.ts} (the component's TypeScript file)
    \item \texttt{component-name.component.html} (the component's HTML template)
    \item \texttt{component-name.component.css} (the component's styles)
    \item \texttt{component-name.component.spec.ts} (the component's unit test file)
\end{itemize}

\section*{Step 4: Add a Service}
Services are used to provide data and functionality in an Angular application.

1. To generate a service, use the following command:
\begin{verbatim}
ng generate service service-name
\end{verbatim}
Or the shorthand:
\begin{verbatim}
ng g s service-name
\end{verbatim}

This will create a service file in \texttt{src/app/service-name.service.ts}.

\section*{Step 5: Add Routes}
To configure routes in Angular:
1. Create an \texttt{app-routing.module.ts} file if it doesn't already exist.
2. Define the routes and import the \texttt{RouterModule}.

Example:
\begin{lstlisting}[language=TypeScript, caption={Defining Routes in Angular}, label={lst:typescript-routes}]
import { NgModule } from '@angular/core';
import { RouterModule, Routes } from '@angular/router';
import { HomeComponent } from './home/home.component';
import { AboutComponent } from './about/about.component';

const routes: Routes = [
  { path: '', component: HomeComponent },
  { path: 'about', component: AboutComponent }
];

@NgModule({
  imports: [RouterModule.forRoot(routes)],
  exports: [RouterModule]
})
export class AppRoutingModule { }
\end{lstlisting}

\section*{Step 6: Deploy the Application}
Once your application is ready, you can build it for production by running the following command:
\begin{verbatim}
ng build --prod
\end{verbatim}

This command will create an optimized version of the application in the \texttt{dist/} folder.

You can then deploy the files from this folder to your production server or a cloud platform such as \texttt{Firebase}, \texttt{AWS}, or \texttt{Netlify}.

\end{document}
